% pendahuluan.tex
\bab{\textit{ABSTRACT}}

\textit{Data that encode the presence of some characteristics typically can be represented
as binary strings. We need similarity functions for binary strings in order to classifycluster them. Existing similarity functions, however, do not take advantage of training
data, which are often available. We believe that a similarity function should be data-
specific. To this end, we use genetic programming (GP) to learn a similarity function from
training data. We propose a novel fitness function that considers five aspects of good
similarity functions, i.e. recall, magnitude, zero-division, identity and symmetry. We also
report mostly-used math operators from extensive literature review. Experiment results
show that GP-based similarity functions outperform the well-known Tanimoto functionmost datasets in terms of classification accuracy using SVMs. In addition, those GP-based
similarity functions are simpler: using less operators and operands. This suggests that our
proposed fitness function for GP is justifiable for learning similarity functions.
}

\vspace{3mm}
\noindent
\textit{\textbf{Keywords:}
binary string, genetic programming, similarity function.
}

